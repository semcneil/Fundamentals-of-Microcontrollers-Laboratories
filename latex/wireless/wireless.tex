\chapter{Wireless}
\chaplabel{wireless}

\section{Purpose}
The goal of this lab is to play with WiFi, Bluetooth, and time. Some of the example
sketches are in a .zip file on Canvas. Download the zip file and unzip it into the 
Arduino directory (typically Documents/Arduino). You should see directories with 
your other sketches inside the Arduino folder.

\section{Laboratory}
\subsection{Preparation}
Before you start this lab, be sure you have updated the following:
\begin{enumerate}
    \item WiFiNINA module firmware (Tools $\rightarrow$ Firmware Updater)
    \item The main libraries used for this lab 
    \begin{enumerate}
        \item WiFiNINA
        \item ArduinoBLE
        \item stm32duino's ST25DV library
    \end{enumerate}
\end{enumerate}

\subsection{WiFi startup}
Follow the WiFiNINA No Encryption example to connect your board to the network. Be 
sure to change the \lstinline|SECRET_SSID| to have \lstinline|EagleNet| between the 
double quotes in the \lstinline|arduino_secrets.h| file.

\subsection{NFC}
The ST25DV library is the one in the Arduino Library Manager written by stm32duino.
The library may show as INCOMPATIBLE even though it works fine. Make sure the 
example (NFC\_Demo.ino) from Canvas works as it is supposed to. It should cycle 
different things (website, call, text, etc.) each time you press right the button. Put 
the top of your phone near the coil that says NFC on the bottom left of the robot.

Change the example to show one of your email addresses and the IP address of your 
Nano RP2040 Connect. This is done by copying over the WiFi connection block from the 
WiFiNINA No Encryption example to your \lstinline|setup()| function. 

In your \lstinline|setup()| function, after WiFi has connected, convert the IPAddress to a 
\lstinline@String@ with the following:\\
\lstinline@IPAddress ip = WiFi.localIP();@\\
\lstinline@String ipString = String(ip[0]) + "." + ip[1] + "." + ip[2] + "." + ip[3];@\\
Now that you have the IP address as a String, you can update the element of \lstinline|NFC_messages| 
that has erau.edu to 
have the IP address String. Also update \lstinline|NFC_protocols| 
to use http NOT https. Note that you have to update an element of the \lstinline|NFC_messages| 
array and it has to be done AFTER the WiFi has been setup. This cannot be done where the array is 
initialized before the \lstinline|setup()| function. 

Demonstrate this to your instructor/TA.


\subsection{UDP Between Boards}
Working with another group, run the examples \lstinline@WiFiUdpSend@ and 
\lstinline@WiFiUdpReceiveSend@ from the .zip file, one on each of 2 boards. 
Be sure to update \lstinline@remoteIp@ in the 
\lstinline@WiFiUdpSend@ sketch to be the IP address of the other board. Once 
running, the right button on the \lstinline@@ board should turn the LED on 
the other board on and off.

Modify this to do something else (buzzer, NeoPixels, motor, etc.). Demo for the 
instructor/TA.

\subsection{Bluetooth Low Energy}
Be sure to generate your own UUIDs for your particular project. If you use
the UUIDs from the examples, you will not be able to tell if you are connecting
to your board or someone else's who is using the same UUIDs. Use the 
\href{https://www.uuidgenerator.net/}{UUID Generator} to create random UUIDs.
Use UUIDs in the form returned by the website. If you create custom ones with 
a different format, the Bluetooth examples fail.
Be sure to set the device (\lstinline@BLE.setDeviceName@) and local names 
(\lstinline@BLE.setLocalName@) on all boards to something unique to 
your group so that you can find it when scanning.

You will need a Bluetooth app for your phone. Some app names to look for are BlueFruit Connect,
LightBlue, and nRF Connect.

\subsubsection{Phone to Robot Interface}
Use the example code (\lstinline@BLE-ButtonLED@) (being sure to change UUIDs, device name, and local name)
to turn an LED on and off on the robot from your phone. Change the code to 
do something else (buzzer, NeoPixel, motor (carefully), servo) when you change 
the value via the Bluetooth interface. Also, show the button on the board changing
the value read on your phone using the Notify property. Demo this for your instructor/TA.

\subsubsection{Inter-Robot Interface}
For this part of the lab you will need to work with another group. Using the example 
of a peripheral (\lstinline@BLE-ButtonLED.ino@) and central (\lstinline@BLE-LED-Central.ino@)
, use one robot to control something on the other robot. Note that you need a different UUID
for each of the peripheral, LED, and Button characteristics but the UUIDs have the be the 
same between the two sketches. Also, the LocalName needs to agree between the sketches. Demo 
this working for your instructor/TA.

% This took difficult library installs and seemed to waste time. 
% update it with instructions about what libraries to install
% \subsection{Clock}
% Using the \lstinline@sd-time-logger.ino@ example for setting the 
% RTC for the SD card, write a program that gets the current time, 
% then displays the time in the format HH:MM:SS and the date 
% (on a different line) in the format YYYY-MM-DD on the display 
% that updates each second. Install the \lstinline@RP2040_RTC@ 
% library with NONE of the other dependencies. 

\subsection{AJAX}
There are 3 files on Canvas:
\begin{enumerate}
    \item AJAX-Robot.ino 
    \item arduino\_secrets.h
    \item index.h 
\end{enumerate}
Open the .ino file and then put the other files in the same directory. Make sure
to add them to the sketch (Sketch$\rightarrow$Add File) so that they show up as separate 
tabs. Upload the file to the robot. Once it has uploaded and the LEDs are green,
put your phone next to the NFC and go to the URL listed (you must be on the 
EagleNet WiFi for it to load). Once you are on the site you can control the 
builtin LED, the servo position, and the NeoPixel color as well as read the 
distance measurement. Demo this working for your instructor/TA.

\section{Shutdown}
Be sure to turn off the battery switch prior to putting the robot away.

\section{Turn In}
Turn in the following:
\begin{enumerate}
    \item Have either the TA or the instructor sign-off on your lab
    \item A PDF of your NFC sketch.
    \item .ino versions of your NFC sketch.
    \item Fill out the end of lab quiz prior to leaving. Note that it includes asking you 
            for the output of the \lstinline$getIDs$ sketch. 
\end{enumerate}

\section{Resources}\label{sec:wirelessresources}
\begin{enumerate}
    \item \href{https://www.uuidgenerator.net/}{UUID Generator} 
    \item PCB Schematic and Layout - see 
            \href{https://github.com/semcneil/Fundamentals-of-Microcontrollers-Manual}{class manual} 
            in the Arduino Startup $\rightarrow$ Schematics and PCB section
\end{enumerate}
This explains some of the code you are using.
\subsubsection{Peripheral Setup}
\begin{enumerate}
    \item Include Arduino's BLE library \\
        \lstinline@#include <ArduinoBLE.h>@ 
    \item Create a BLEService. This is container for characteristics. You can
            have multiple services and characteristics per service.\\
        \lstinline@BLEService ledService(String UUID);@
    \item Create a variable for each characteristic. Characteristics have properties. 
            The most used properties are:
        \begin{enumerate}
            \item BLERead - This allows a connected device to read the value of this variable
            \item BLEWrite - This allows a connected device to change the value of the variable
            \item BLENotify -  This allows a connected device to be notified if the value changes
        \end{enumerate}
        \lstinline@BLEByteCharacteristic LEDCharacteristic(String UUID, BLERead | BLEWrite);@ \\
        \lstinline@BLEByteCharacteristic buttonCharacteristic(String UUID, BLERead | BLENotify);@
    \item The next setup is done in the \lstinline@setup()@ function
    \item Start the BLE module \\
        \lstinline@BLE.begin()@ 
    \item Set the device name. This is the externally advertised name \\
        \lstinline@BLE.setDeviceName("BUTTON_LED");@ 
    \item Set the local name. This can be checked by a device once connected \\
        \lstinline@BLE.setLocalName("BUTTON_LED_MCNEILL");@ 
    \item Set the service as advertised \\
        \lstinline@BLE.setAdvertisedService(ledService);@
    \item Add the characteristics to the service \\
        \lstinline@ledService.addCharacteristic(LEDCharacteristic);@
    \item Add the service to the BLE object \\
        \lstinline@BLE.addService(ledService);@
    \item Set initial values for each characteristic \\
        \lstinline@LEDCharacteristic.writeValue(0);@
    \item Finish by starting your BLE advertising its existence \\
        \lstinline@BLE.advertise();@
    \item In the \lstinline@loop()@ my preferred method is to create central device
            and check to see if a device has connected. Once it has, update the 
            values of the output characteristics and check if the writeable 
            characteristics have been written to.
\end{enumerate}
\subsubsection{Central Setup}
\begin{enumerate}
    \item Scan for a device, typically by service UUID. \\
        \lstinline@BLE.scanForUuid(BLE_UUID_PERIPHERAL);@ 
    \item Create a peripheral object \\
        \lstinline@BLEDevice peripheral = BLE.available();@ 
    \item If the peripheral has been discovered, check its local name  \\
        \lstinline@if (peripheral.localName() != "BUTTON_LED")@ 
    \item Stop scanning \\
        \lstinline@BLE.stopScan();@ 
    \item Connect to the peripheral \\
        \lstinline@peripheral.connect()@ 
    \item Read the peripheral's attributes \\
        \lstinline@peripheral.discoverAttributes()@ 
    \item Retrieve each characteristic and check to make sure each has the 
            attributes (Read/Write/Notify) expected
        \begin{lstlisting}
if (!buttonCharacteristic) {
    Serial.println("Peripheral does not have button characteristic!");
    peripheral.disconnect();
    return;
} else if (!buttonCharacteristic.canRead()) {
    Serial.println("Peripheral does not have a readable button characteristic!");
    peripheral.disconnect();
    return;
} else if (!buttonCharacteristic.canSubscribe()) {
    Serial.println("Peripheral does not allow button subscriptions (notify)");
} else if(!buttonCharacteristic.subscribe()) {
    Serial.println("Subscription failed!");
} else {
    Serial.println("Connected to button characteristic");
}
        \end{lstlisting}
    \item Read/write the peripheral's characteristics as desired
        \begin{lstlisting}
if(buttonCharacteristic && buttonCharacteristic.canRead() && buttonCharacteristic.valueUpdated()) {
    byte peripheralButtonState;
    buttonCharacteristic.readValue(peripheralButtonState);
    if(!peripheralButtonState) {
        digitalWrite(ledPin, HIGH);
    } else {
        digitalWrite(ledPin, LOW);
    }
}          
        \end{lstlisting}
\end{enumerate}

