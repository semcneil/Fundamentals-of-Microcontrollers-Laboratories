\chapter{Wireless}
\chaplabel{wireless}

\section{Purpose}
The goal of this lab is to play with WiFi, Bluetooth, and time. Some of the example
sketches are in a .zip file on Canvas. Download the zip file and unzip it into the 
Arduino directory (typically Documents/Arduino). You should see directories with 
your other sketches inside the Arduino folder.

\section{Laboratory}
\subsection{NFC}
The ST25DV library is the one in the Arduino Library Manages written by stm32duino.
Use the example to make sure that your NFC chip is updated with the latest
IP address for your module (this requires adding WiFi connection to your sketch), 
and one of your email addresses. Test this and as with each other section, 
demonstrate this to your instructor/TA.

\subsection{UDP Between Boards}
Run the examples \lstinline@WiFiUdpSend@ and \lstinline@WiFiUdpReceiveSend@, one 
on each of 2 boards. Once running, the right button on the \lstinline@@ board 
should turn the LED on the other board on and off.

Modify this to do something else (buzzer, NeoPixels, motor, etc.). Demo for the 
instructor/TA.

\subsection{Bluetooth Control of the Robot}
Be sure to generate your own UUIDs for your particular project. If you use
the UUIDs from the examples, you will not be able to tell if you are connecting
to your board or someone else's who is using the same UUIDs. You can use the 
\href{https://www.uuidgenerator.net/}{UUID Generator} to create random UUIDs.
Be sure to set the device (\lstinline@BLE.setDeviceName@) and local names 
(\lstinline@BLE.setLocalName@) on all boards to something unique to 
your group so that you can find it when scanning.

\subsubsection{Phone to Robot Interface}
Use the example code (being sure to change UUID, device name and local name)
to turn an LED on and off on the robot from your phone. Change the code to 
do something else (buzzer, NeoPixel, motor (carefully), servo) when you change 
the value via the Bluetooth interface. 

\subsubsection{Inter-Robot Interface}
For this part of the lab you will need to work with another group. Using the example 
of a peripheral and central, use one robot to control something on the other robot.

\subsection{Clock}
Using the example for setting the RTC for the SD card, write a 
program that gets the current time, then displays the time in the 
format HH:MM:SS and the date (on a different line) in the format 
YYYY-MM-DD on the display that updates each second.

\section{Shutdown}
Be sure to turn off the battery switch prior to putting the robot away.

\section{Turn In}
Turn in the following:
\begin{enumerate}
    \item Have either the TA or the instructor sign-off on your lab
    \item A PDF of your sketch.
    \item .ino versions of your sketch.
    \item Fill out the end of lab quiz prior to leaving. Note that it includes asking you 
            for the output of the \lstinline$getIDs$ sketch. 
\end{enumerate}

\section{Resources}%\label{sec:distmotorservoresources}
\begin{enumerate}
    \item \href{https://www.uuidgenerator.net/}{UUID Generator} 
    \item PCB Schematic and Layout - see 
            \href{https://github.com/semcneil/Fundamentals-of-Microcontrollers-Manual}{class manual} 
            in the Arduino Startup $\rightarrow$ Schematics and PCB section
\end{enumerate}

