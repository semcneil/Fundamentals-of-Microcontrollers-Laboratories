\chapter{IMU}
\chaplabel{imu}

\section{Purpose}
The goal of this lab is to learn to use the IMU data and the SD card.

\subsection{IMU Angle Measurement\label{subsec:imuangle}}
The \href{https://semcneil.github.io/Fundamentals-of-Microcontrollers-Manual/Fundamentals-of-Microcontrollers.pdf}{user manual} 
has the equations for calculating the current angle of the IMU on one axis
using a complementary filter to combine data from the gyroscope and the accelerometers. The 
equation is copied here as Equation \ref{eq:anglefused}. 

\begin{equation}
    \label{eq:anglefused}
    \theta_{mixed}[t] = \alpha \left(\theta_{mixed}[t-1] + \omega_{gyro}\Delta t \right) + (1-\alpha)\atan2(a_x,a_z)
\end{equation}

\section{Main Requirements}
Write two sketches with the following functionality:

\subsection{IMU Screen Control}
Use the angle measurement to move a character on the screen as described in the Procedure section.

\subsection{Plotting IMU Data}
Save the IMU data as specified in the Procedure to the SD card, then plot the saved data in Matlab 
or your other favorite plotting program.

\section{Procedure}
It is best to follow this outline to finish this lab most efficiently.
\begin{enumerate}
    \item Start by making sure the \lstinline@IMU-CompFilterEx.ino@ example on Canvas works as it should for you.
    \item Make it so that the angle controls the movement of a character on the screen. You can use the ideas 
            from the programs you and your partner made for \hyperref[chap:displays]{Lab \ref*{chap:displays}}.
    \begin{enumerate}
        \item -180 degrees puts the symbol all the way to the right
        \item 0 puts the symbol in the middle of the screen
        \item +180 degrees puts the symbol all the way to the left
    \end{enumerate}
    \item As the second part of the lab, load the example named \lstinline@SDReadWrite@ 
    \item Run it to make sure it works as advertised.
    \item NOTE: File names have to be 8.3 -- maximum of 8 characters before the . and 3 after it.
    \item Change it to save \lstinline@t@ (from curTime), \lstinline@ax@, \lstinline@az@, 
            \lstinline@theta_g@, \lstinline@theta_a@, and \lstinline@theta@ to the SD card 
    \item Delete the file reading operations and make sure that the first write truncates and 
            subsequent ones do not.
    \item Notice what \lstinline@dt@ is now.
    \item Make of plot of the data in Matlab, Python, or your other favorite program for making good plots.
\end{enumerate}

\section{Extra Credit: 2~pts}
After you have completed all the lab requirements, make it so that the character on the screen moves
up and down as well as left and right based on the other axis of the PCB rotation.

\section{Turn In}
Turn in the following:
\begin{enumerate}
    \item Have either the TA or the instructor sign-off on your lab
    \item PDFs of your sketches and your plots of the IMU data.
    \item .ino versions of your sketch.
    \item The script/file you used to plot the data.
    \item Fill out the end of lab quiz prior to leaving. Note that it includes asking you 
            for the output of the \lstinline$getIDs$ sketch. 
\end{enumerate}

\section{Resources}\label{sec:imuresources}
\begin{enumerate}
    \item The \href{https://www.st.com/resource/en/datasheet/lsm6dsox.pdf}{LSM6DSOX IMU datasheet}
    \item PCB Schematic and Layout - see 
            \href{https://github.com/semcneil/Fundamentals-of-Microcontrollers-Manual}{class manual} 
            in the Arduino Startup $\rightarrow$ Schematics and PCB section
\end{enumerate}

