\chapter{Displays}
\chaplabel{displays}

\section{Purpose}
The goal of this lab is to learn to use the TFT display. Debouncing the buttons 
so that an \lstinline|if| statement only executes once per press takes the most time.

\begin{lstlisting}[language=C++, caption={This is example code for debouncing a button.},label={lst:debounceEx}]
int lastButtonVal = 1;

void loop() {
    int curButtonVal = digitalRead(RIGHT_BUTTON_PIN);
    if((lastButtonVal != curButtonVal) && !curButtonVal) {
        // do something when button pressed
    }
    lastButtonVal = curButtonVal; // very important this is outside the if and inside loop()
}
\end{lstlisting}

There is code on Canvas that demonstrates some display capabilities, but do NOT use the
\lstinline|rButtonWait()| function.

\section{Procedure}
\subsection{Main Requirements}
Write a sketch with the following functionality:
\begin{enumerate}
    \item Display a custom message for 3~seconds indicating the start of the program when
            your program starts 
    \item Choose a favorite character and have it move left 10~pixels for each press
            of the left button and right 10~pixels for each press of the right button.
    \begin{enumerate}
        \item Choose the Y value to be one that looks good to you
        \item If 10~pixels seems to not be a good value feel free to change it, just note
                that you changed the value
        \item This builds on last week's making all the buttons function
        \item Debounce the buttons so that each press only moves the character once
        \item Add logic so that the character doesn't go far off the edge of the screen
                on either side.
    \end{enumerate}
    \item Draw some (at least 3, but they don't all have to be different) of the 
            drawing primitives (line, circle, rectangle, etc.)
\end{enumerate}

\subsection{Extra Credit}
The following are extra credit options with increasing value:
\begin{enumerate}
    \item Make the front and back switches move your character up and down along 
            with the left and right from the Main Requirements. (1~point)
    \item Make an animation of some kind that lasts at least two seconds. (1~points)
    \item Make a game that uses the buttons and the screen (2~points)
\end{enumerate}

\section{Turn In}
Turn in the following:
\begin{enumerate}
    \item Make sure that the TA/Instructor signs off on your sketch demonstration.
    \item A PDF of your sketch.
    \item .ino versions of your sketch.
    \item Fill out the end of lab quiz prior to leaving. Note that it includes asking you 
            for the output of the \lstinline$getIDs$ sketch. 
\end{enumerate}

\section{Resources}\label{sec:displaysresources}
\begin{enumerate}
    \item \href{https://www.adafruit.com/product/4383}{Adafruit 1.14" 240x135 Color TFT Display + MicroSD Card Breakout - ST7789}
    \item \href{https://www.arduino.cc/reference/en/libraries/adafruit-st7735-and-st7789-library/}{Adafruit ST7789 library}
    \item \href{https://learn.adafruit.com/adafruit-gfx-graphics-library}{Adafruit GFX library}
    \item See the chapter in the \href{https://github.com/semcneil/Fundamentals-of-Microcontrollers-Manual}{class manual} about displays
    \item PCB Schematic and Layout - see 
            \href{https://github.com/semcneil/Fundamentals-of-Microcontrollers-Manual}{class manual} 
            in the Arduino Startup $\rightarrow$ Schematics and PCB section
\end{enumerate}

