\chapter{Environmental Sensing}
\chaplabel{envSensing}

\section{Purpose}
The goal of this lab is to learn to collect data with an ADC and other sensors.

\section{Procedure}
\subsection{Overview}
Write a sketch with the following functionality:

Collect the following data and display it once a second via serial and on the display with the 
appropriate labels and units if you can make them fit.
\begin{enumerate}
	\item The output from \lstinline@millis()@ (ms)
	\item Both light intensities as a number between 0 and 1023 (unitless)
	\item Potentiometer value as a voltage between 0 and 3.3~V
	\item Temperature in Fahrenheit from TEMP0 (U27 attached to input 0 of U37)
    \item Get SHT31 reporting temperature in F and humidity via Serial
    \item Add SHT31 data to display 
    \item Get compass (QMC5883L) reporting azimuth via Serial
    \item Add azimuth data to the display
    \item Get IMU (LSM6DSOX) reporting accelerations and gyroscope data via Serial 
    \item Add accelerometer and gyroscope data to display
    % \item Get 1-Wire (DS18B20) sensor(s) reporting temperature(s) via Serial
    % \item Display 1-Wire temperature(s) on display
\end{enumerate}


\subsection{Suggestions}
Use a \href{https://www.arduino.cc/reference/en/language/variables/data-types/stringobject/}{String} 
object to accumulate your display string and then call \lstinline@display.println(yourString)@ 
to display it. Note a few things:
\begin{enumerate}
	\item The String type starts with a capital S.
	\item You can add to the String object using \lstinline@+=@ or just \lstinline@+@, 
		but with only the plus operator, all arguments have to be of the same type. 
	\item The String object also allows you to limit the number of decimal places for \lstinline@float@ types.
            \lstinline@String(tF,1)@ displays tF to 1 decimal place.
\end{enumerate}
An example is shown in Listing \ref{lst:dispstr}.
\begin{lstlisting}[caption={This is an example of using a String 
		object to display text and float variables. The floats are 
		limited to 1 decimal place such that 7.123 would be displayed as 7.1.},
		label={lst:dispstr},language=C++]
	String dispStr;
	dispStr = "T(F): ";
	dispStr += String(tF,1);
	dispStr += "\n";
	// Control the display  
	tft.fillScreen(ST77XX_BLACK);  // clear display
	tft.setTextColor(ST77XX_YELLOW);  // set text color
	tft.setTextSize(1);  // Normal 1:1 pixel scale
	tft.setCursor(0,0);  // Start at top-left corner
	tft.println(dispStr);
\end{lstlisting}
However, if you want to have different portions of the text in different colors you will
need to call \lstinline@tft.setTextColor@ between each \lstinline@tft.println@.

\subsection{ADS7142 - TEMP0, POT, LIGHT1, LIGHT2}
The light sensors, potentiometer, and temperature sensor are connected to ADS7142 
analog-to-digital (ADC) sensors. There is a library in the normal library adding 
method. Look for the library for the ADS7142 named as being by either Seth McNeill 
or Anitracks. Different versions of the IDE show different names for the same library.
After you install it,
there should now be examples under Anitracks ADS7142. Follow the read2Ch example 
combined with your previous work to complete this lab. Note, that there are two 
ADS7142s on the board and each has a different address and two inputs for a total 
of 4 inputs. Check the schematics in the class manual to know which address goes 
to which inputs.

The light sensors are CdS light reactive resistors setup as resistor dividers. They 
are located in the bottom left and right corners of the circuit board and look like 
roundish gray with red lines on them. We tried putting mirrors over some of them to 
try using them as distance sensors, but that didn't work well so many of the mirrors 
have broken off leaving some hot glue residue.

For converting to voltage, note that the ADS7142 is a 12-bit ADC but returns values 
as 16-bit which means that the maxADCValue is 65535. The reference voltage is 3.3~V.

As a contrast, if you read the voltage from A6 and A7 and use analogRead, the 
maxADCValue is 1023 since they return 10-bit values and the reference voltage 
is 1.1~V.

\subsection{SHT31 Temperature and Humidity Sensor}
Use Adafruit's SHT31 library. Base your code on their example. Be sure 
to make sure that the heater is off. For this class, you do not need to 
turn on the heater.

Also, the SHT31 needs to be reset each time the program starts. It's 
reset pin is connected to a PCA9535. An example of how to do this
is shown in Listing \ref{lst:sht31reset}.

\begin{lstlisting}[caption={This listing shows how to reset the SHT31.},
label={lst:sht31reset},language=C++]
Wire.begin();
muxU31.attach(Wire, 0x20);
muxU31.polarity(PCA95x5::Polarity::ORIGINAL_ALL);
muxU31.direction(0x1CFF);  // 1 is input, see schematic to get upper and lower bytes
muxU31.write(PCA95x5::Port::P10, PCA95x5::Level::L);  // disable SHT31
delay(100);
muxU31.write(PCA95x5::Port::P10, PCA95x5::Level::H);  // enable SHT31
delay(100);

Serial.println("SHT31 test");
if (! sht31.begin(0x44)) {   // Set to 0x45 for alternate i2c addr
    Serial.println("Couldn't find SHT31");
    while (1) delay(1);
}
\end{lstlisting}

\subsection{QMC5883L Compass}
Use the library by MPrograms for the QMC5883L compass. The azimuth example 
should provide what you need to make it go.

\subsection{LSM6DSOX IMU}
Install the Arduino library for the LSM6DSOX IMU. Your program will need 
to use a combination of the SimpleAccelerometer and SimpleGyroscope examples.
NOTE: This library returns acceleration in g's not $m/s^2$.

\subsection{(Extra Credit) 1-Wire Sensors}
As mentioned in class, the only library I have found to work with the Nano Connect 
RP2040 is the \href{https://github.com/pstolarz/OneWireNg}{OneWireNg} library. This 
library has to be of a version less than 0.13 as of 2023 September 21.
Unfortunately, the library examples are very complex. There is a simpler solution by 
combining the examples from the OneWire library with the capabilities of the OneWireNg 
library. It has to be done carefully though. Here is how to do it:
\begin{enumerate}
    \item In the Arduino IDE install both the OneWire and OneWireNg libraries
    \item Open the OneWire example named \lstinline@DS18x20_Temperature@ 
    \item Add the following \lstinline|#include| to the very top of the example file:\\
            \lstinline@#include "OneWireNg_CurrentPlatform.h"@ 
    \item This example should now compile and run just fine 
\end{enumerate}

I suggest putting the OneWire example in its own function that returns the temperature of 
interest then call that function in your loop.

\subsubsection{NOTE}
This example reads the results from one (1) sensor per pass through the 
\lstinline@loop()@ function. For your program, you will want to collect 
all three sensors' data in a single pass through the \lstinline@loop()@ 
function. 

Start by just reading one DS1820 then after you have everything
else working, come back and add in the rest of the DS1820 sensors.

\section{Main Requirements}
Write a sketch with the following functionality:

Collect the following data and display it once a second via serial and on 
the display with the appropriate labels and units if you can make them fit.
\begin{enumerate}
	\item The output from \lstinline@millis()@ (ms)
	\item Both light intensities as a number between 0 and 1023 (unitless)
	\item Potentiometer value as a voltage between 0 and 3.3~V
	\item Temperature in Fahrenheit from TEMP0 (U27 attached to input 0 of U37)
	\item The temperature in Fahrenheit and humidity from the SHT31 sensor
	\item The compass azimuth angle
	\item The 3-axis accelerometer and gyroscope data (6 data points)
 	\item (Extra Credit) The temperatures from at least one 1-Wire DS18B20 sensor (can take about
            seconds for acquisition for 3 sensors). Do this part last.
\end{enumerate}

\section{Turn In}
Turn in the following:
\begin{enumerate}
    \item Have either the TA or the instructor sign-off on your lab
    \item A PDF of your sketch.
    \item .ino versions of your sketch.
    \item Fill out the end of lab quiz prior to leaving. Note that it includes asking you 
            for the output of the \lstinline$getIDs$ sketch. 
\end{enumerate}

\section{Resources}\label{sec:datacollectionsresources}
\begin{enumerate}
    \item \href{https://www.adafruit.com/product/4383}{Adafruit 1.14" 240x135 Color TFT Display + MicroSD Card Breakout - ST7789}
    \item \href{https://www.arduino.cc/reference/en/libraries/adafruit-st7735-and-st7789-library/}{Adafruit ST7789 library}
    \item \href{https://learn.adafruit.com/adafruit-gfx-graphics-library}{Adafruit GFX library}
    \item The \href{https://sensirion.com/products/catalog/SHT31-DIS-B/}{SHT31 temperature and humidity sensor datasheet}
    \item The \href{https://www.filipeflop.com/img/files/download/Datasheet-QMC5883L-1.0%20.pdf}{QMC5883L compass datasheet}
    \item The \href{https://www.st.com/resource/en/datasheet/lsm6dsox.pdf}{LSM6DSOX IMU datasheet}    \item See the chapter in the \href{https://github.com/semcneil/Fundamentals-of-Microcontrollers-Manual}{class manual} about displays
    \item The \href{https://cdn-shop.adafruit.com/datasheets/DS18B20.pdf}{DS18B20 temperature sensor datasheet}
    \item PCB Schematic and Layout - see 
            \href{https://github.com/semcneil/Fundamentals-of-Microcontrollers-Manual}{class manual} 
            in the Arduino Startup $\rightarrow$ Schematics and PCB section
\end{enumerate}

