\chapter{More Sensors}
\chaplabel{moreSensors}

\section{Purpose}
The goal of this lab is to learn to collect data with more sensors.

\section{Procedure}
It is best to follow this outline to finish this lab most efficiently.
\begin{enumerate}
    \item Get 1-Wire (DS18B20) sensors reporting temperatures via Serial
    \item Display 1-Wire temperatures on display
    \item Get SHT31 reporting temperature and humidity via Serial
    \item Add SHT31 data to display 
    \item Get compass (QMC5883L) reporting azimuth via Serial
    \item Add azimuth data to the display
    \item Get IMU (LSM6DSOX) reporting accelerations and gyroscope data via Serial 
    \item Add accelerometer and gyroscope data to display
\end{enumerate}

\subsection{1-Wire Sensors}
As mentioned in class, the only library I have found to work with the Nano Connect 
RP2040 is the \href{https://github.com/pstolarz/OneWireNg}{OneWireNg} library. 
Unfortunately, the library examples are very complex. There is a simpler solution by 
combining the examples from the OneWire library with the capabilities of the OneWireNg 
library. It has to be done carefully though. Here is how to do it:
\begin{enumerate}
    \item In the Arduino IDE install both the OneWire and OneWireNg libraries
    \item Open the OneWire example named \lstinline@DS18x20_Temperature@ 
    \item Add this line above the \lstinline@#include <OneWire.h>@ line:\\
            \lstinline@#include "OneWireNg_CurrentPlatform.h"@ 
    \item This example should now compile and run just fine 
\end{enumerate}

\subsubsection{NOTE}
This example reads the results from one (1) sensor per pass through the 
\lstinline@loop()@ function. For your program, you will want to collect 
all three sensors' data in a single pass through the \lstinline@loop()@ 
function.

\subsection{SHT31 Temperature and Humidity Sensor}
Use Adafruit's SHT31 library. Base your code on their example. Be sure 
to make sure that the heater is off. For this class, you do not need to 
turn on the heater.

\subsection{QMC5883L Compass}
Use the library by MRPrograms for the QMC5883L compass. The azimuth example 
should provide what you need to make it go.

\subsection{LSM6DSOX IMU}
Install the Arduino library for the LSM6DSOX IMU. Your program will need 
to use a combination of the SimpleAccelerometer and SimpleGyroscope examples.

\section{Main Requirements}
Write a sketch with the following functionality:

Collect the following data and display it once a second on the display with the 
appropriate labels and units if you can make them fit.
\begin{enumerate}
	\item The temperatures from all 3 1-Wire DS18B20 sensors
	\item The temperature and humidity from the SHT31 sensor
	\item The compass azimuth angle
	\item The 3-axis accelerometer and gyroscope data (6 data points)
\end{enumerate}


\section{Turn In}
Turn in the following:
\begin{enumerate}
    \item A video of your board fulfilling the requirements in the Procedure section 
            that includes both of your faces and you saying your names.
    \item A PDF of your sketch.
    \item .ino versions of your sketch.
    \item Fill out the end of lab quiz prior to leaving. Note that it includes asking you 
            for the output of the \lstinline$getIDs$ sketch. 
\end{enumerate}

\section{Resources}\label{sec:datacollectionsresources}
\begin{enumerate}
    \item The \href{https://cdn-shop.adafruit.com/datasheets/DS18B20.pdf}{DS18B20 temperature sensor datasheet}
    \item The \href{https://sensirion.com/products/catalog/SHT31-DIS-B/}{SHT31 temperature and humidity sensor datasheet}
    \item The \href{https://www.filipeflop.com/img/files/download/Datasheet-QMC5883L-1.0%20.pdf}{QMC5883L compass datasheet}
    \item The \href{https://www.st.com/resource/en/datasheet/lsm6dsox.pdf}{LSM6DSOX IMU datasheet}
    \item PCB Schematic and Layout - see 
            \href{https://github.com/semcneil/Fundamentals-of-Microcontrollers-Manual}{class manual} 
            in the Arduino Startup $\rightarrow$ Schematics and PCB section
\end{enumerate}

