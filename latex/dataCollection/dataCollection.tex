\chapter{Data Collection}
\chaplabel{dataCollection}

\section{Purpose}
The goal of this lab is to learn to collect data with an ADC and other sensors.

\section{Procedure}
\subsection{Main Requirements}
Write a sketch with the following functionality:

Collect the following data and display it once a second on the display with the 
appropriate labels and units if you can make them fit.
\begin{enumerate}
	\item The output from \lstinline@millis()@ (ms)
	\item Both light intensities as a number between 0 and 1023 (unitless)
	\item Potentiometer value as a voltage between 0 and 3.3~V
	\item Temperature in Fahrenheit from TEMP0 (U27 attached to input 0 of U37)
\end{enumerate}
The light sensors, potentiometer, and temperature sensor are connected to ADS7142 
analog-to-digital (ADC) sensors. As of 2022 October 03, there is not a library in 
the Arduino system for this chip, so you will to use the 
\href{https://github.com/semcneil/ADS7142_Arduino_Library}{one written by semcneil}.
Use the usual process:
\begin{enumerate}
    \item Click on the green Code button
    \item Download the Zip file 
    \item Go to the Arduino IDE
    \item Sketch $\rightarrow$ Include Library $\rightarrow$ Add .ZIP library 
\end{enumerate}
When you go back to File $\rightarrow$ Examples, there should now be an example under 
Anitracks ADS7142. Follow the example combined with your previous work to complete this
lab.

\subsection{Suggestions}
Use a \href{https://www.arduino.cc/reference/en/language/variables/data-types/stringobject/}{String} 
object to accumulate your display string and then call \lstinline@display.println(yourString)@ 
to display it. Note a few things:
\begin{enumerate}
	\item The String type starts with a capital S.
	\item You can add to the String object using \lstinline@+=@ or just \lstinline@+@, 
		but with only the plus operator, all arguments have to be of the same type. 
	\item The String object also allows you to limit the number of decimal places for \lstinline@float@ types.
            \lstinline@String(tF,1)@ displays tF to 1 decimal place.
\end{enumerate}
An example is shown in Listing \ref{lst:dispstr}.
\begin{lstlisting}[caption={This is an example of using a String 
		object to display text and float variables. The floats are 
		limited to 1 decimal place such that 7.123 would be displayed as 7.1.},
		label={lst:dispstr},language=C++]
	String dispStr;
	dispStr = "T(F): ";
	dispStr += String(tF,1);
	dispStr += "\n";
	// Control the display  
	tft.fillScreen(ST77XX_BLACK);  // clear display
	tft.setTextColor(ST77XX_YELLOW);  // set text color
	tft.setTextSize(1);  // Normal 1:1 pixel scale
	tft.setCursor(0,0);  // Start at top-left corner
	tft.println(dispStr);
\end{lstlisting}
However, if you want to have different portions of the text in different colors you will
need to call \lstinline@tft.setTextColor@ between each \lstinline@tft.println@.

\section{Turn In}
Turn in the following:
\begin{enumerate}
    \item A video of your board fulfilling the requirements in the Procedure section 
            that includes both of your faces and you saying your names.
    \item A PDF of your sketch.
    \item .ino versions of your sketch.
    \item Fill out the end of lab quiz prior to leaving. Note that it includes asking you 
            for the output of the \lstinline$getIDs$ sketch. 
\end{enumerate}

\section{Resources}\label{sec:datacollectionsresources}
\begin{enumerate}
    \item \href{https://www.adafruit.com/product/4383}{Adafruit 1.14" 240x135 Color TFT Display + MicroSD Card Breakout - ST7789}
    \item \href{https://www.arduino.cc/reference/en/libraries/adafruit-st7735-and-st7789-library/}{Adafruit ST7789 library}
    \item \href{https://learn.adafruit.com/adafruit-gfx-graphics-library}{Adafruit GFX library}
    \item See the chapter in the \href{https://github.com/semcneil/Fundamentals-of-Microcontrollers-Manual}{class manual} about displays
    \item PCB Schematic and Layout - see 
            \href{https://github.com/semcneil/Fundamentals-of-Microcontrollers-Manual}{class manual} 
            in the Arduino Startup $\rightarrow$ Schematics and PCB section
\end{enumerate}

