\chapter{Machine Learning}
\chaplabel{machineLearning}

\section{Purpose}
The Arduino Nano Connect RP2040 has the LSM6DSOX IMU made by ST Microelectronics. It has 
a Machine Learning Core inside. It also has some preset systems inside for detecting 
the following:
\begin{enumerate}
    \item 6D orientation (up, down, left, right, front, back)
    \item Free fall detection 
    \item Pedometer (counting steps)
    \item Tap detection, both single and double tap 
    \item Tilt detection (for orienting screens as users rotate the device)
    \item Wake up detection to let the device (microcontroller) know the device has moved
\end{enumerate}
Today's lab will run each of the built-in systems plus a pretrained ML model.

\section{Laboratory}
\subsection{Download Examples}
There is a zip file on Canvas named LSM6DSOX\_Examples.zip. Download and extract the zip 
file. Move the resulting folder into your sketch directory (usually named arduino or Arduino 
inside your Documents folder). Now try opening one of the examples from within the Arduino IDE.
If you can find it in your File $\rightarrow$ Sketchbook menu, you have put them in the right
place.

\subsection{Install Library}
Install the STM32duino\_LSM6DSOX library in the Arduino IDE since these examples rely on 
it for the interface to the IMU.

\subsection{Running Examples}
Run each of the following examples noting that the outputs will all be via the serial port:

\subsubsection{6D Orientation}
Open the example named LSM6DSOX\_6DOrientation. Compile and upload it to the board. When it
runs, it should output a drawing indicating the orientation as you rotate the board.

\subsubsection{Free Fall Detection}
Don't drop or throw the board! Open the example named LSM6DSOX\_FreeFallDetection. Compile 
and upload it to your board. When it is running, raise and lower the board while holding it 
and it should send an output when it thinks it is in free fall. 

\subsubsection{Pedometer}
Open the example named LSM6DSOX\_Pedometer. Compile and upload it to the board. Once it is 
running it should print out the number of steps periodically. Shake the board up and down 
to imitate walking and it should increment the counter.

\subsubsection{Tap Detection}
Open the example named LSM6DSOX\_TapDetect. Compile and upload it to the board. Once it is 
running try tapping the side of the board/module (not front, back, or top). The serial should 
output single tap and double tap as it thinks it is tapped.

\subsubsection{Tilt Detection}
Open the example named LSM6DSOX\_TiltDetection. Compile and upload it to the board. Once it is 
running try tilting the board after letting it sit in one orientation for a bit. It should 
trigger a serial output when you move the board.

\subsubsection{Wake Up Detection}
Open the example named LSM6DSOX\_WakeUpDetection. Compile and upload it to the board. Once it is 
running try moving the board after letting it sit still for a bit. It should trigger a serial 
output when you move the board.

\subsubsection{Machine Learning Example}
Open the example named LSM6DSOX\_MLC. Compile and upload it to the board. Once it is 
running try moving the board to simulate walking, running, biking, driving, unknown, and 
staying stationary. Walking and jogging are the easiest to get with vertical motion (Z-axis). 
Biking seems to be lateral motion (X- or Y-axis). I have seen driving once but don't remember 
how I got it. I have never seen unknown.

\subsection{Using the Examples}
Take one or more of the examples and do something with them. It could be as simple as 
adding sound or screen output to the example(s) or something more.

\medskip

\noindent Have fun!

\section{Turn In}
Turn in the following:
\begin{enumerate}
    \item Have either the TA or the instructor sign-off on your lab
    \item A PDF of your sketch.
    \item .ino versions of your sketch.
    \item Fill out the end of lab quiz prior to leaving. Note that it includes asking you 
            for the output of the \lstinline$getIDs$ sketch. 
\end{enumerate}

\section{Resources}%\label{sec:distmotorservoresources}
\begin{enumerate}
    \item \href{https://www.st.com/resource/en/datasheet/lsm6dsox.pdf}{LSM6DSOX Datasheet}
    \item PCB Schematic and Layout - see 
            \href{https://github.com/semcneil/Fundamentals-of-Microcontrollers-Manual}{class manual} 
            in the Arduino Startup $\rightarrow$ Schematics and PCB section
\end{enumerate}

