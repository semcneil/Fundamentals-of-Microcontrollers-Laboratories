\chapter{Multiplexer, LED Display, Binary, HEX}
\chaplabel{muxled}

\section{Purpose}
The goal of this lab is to use binary and hexadecimal numbers in an applied setting. This is 
achieved by having you (the student) use the Arduino Nano Connect RP2040 to setup the 
PCA9535 I$^2$C Input/Output (IO) port chip to drive a 4-digit LED display.

\section{Resources}\label{sec:resources}
\begin{enumerate}
    \item \href{https://www.nxp.com/products/interfaces/ic-spi-i3c-interface-devices/general-purpose-i-o-gpio/16-bit-ic-bus-and-smbus-low-power-i-o-port-with-interrupt:PCA9535_PCA9535C}{PCA9535 datasheet}
    % \item \href{https://datasheet.lcsc.com/lcsc/1810191633_SUNLIGHT-SLR0394FG3C5BD-3-5_C225905.pdf}{LED display datasheet}
    \item See schematics in the class manual for LED wiring.
 %   \item \href{https://github.com/semcneil/PCA95x5}{PCA9535 library} (McNeill version)
    \item PCB Schematic and Layout - see 
            \href{https://github.com/semcneil/Fundamentals-of-Microcontrollers-Manual}{class manual} 
            in the Arduino Startup $\rightarrow$ Schematics and PCB section
\end{enumerate}

\section{Procedure}
\subsection{Add the PCA95x5 library}
In Sketch $\rightarrow$ Include Library $\rightarrow$ Manage Libraries search for 
PCA9535 and install the latest version of the one by hideakitai.
% Normally libraries can be added by using the Arduino IDE library manager. However, this 
% library hasn't been included in it yet, so you will install it by hand. This is an 
% important skill to know since not all the software you will need to use is in the 
% normal library system.

% \begin{enumerate}
%     \item Open the Arduino IDE 
%     \item Open preferences
%     \item Note the Sketchbook location 
%     \item Go to the website for the PCA9535 library (see \ref{sec:resources} Resources section)
%     \item Click the green Code button and then Download ZIP
%     \item Save the zip file to your Downloads folder or somewhere else you can find it again
%     \item In the Arduino IDE select Sketch $\rightarrow$ Include Library $\rightarrow$ Add .ZIP Library
%     \item Select the zip file you downloaded earlier
%     \item Wait for the IDE to say ``Library added to your libraries"
%     \item Check that the library is indeed installed one of two ways:
%     \begin{enumerate}
%         \item Go to Sketch $\rightarrow$ Include Libraries and look for PCA95x5
%         \item Go to File $\rightarrow$ Examples and look for the PCA95x5 examples
%     \end{enumerate}
%     \item The PCA95x5 library is now added successfully.
% \end{enumerate}

\subsection{Turn on some LEDs}
An example script is shown in Listing \ref{lst:leddispstart}. Use this as a starting point for your
code. Be sure to change the header information to include all lab partner names and the correct 
date. This script displays an 8. in the first digit and zeros in the rest. Note the clearing
lines to make it so that there is not ghosting of numbers to the right of where they are displayed. 
Remember that we are using the PCA9535 chip which requires the PCA95x5 library. When using a library 
the following steps must be followed or else the microcontroller will probably crash or at least 
the part you are trying to use will not work. Note the steps in Listing \ref{lst:leddispstart}.

\begin{enumerate}
    \item Include the library: \lstinline|#include <PCA95x5.h>|
    \item Create an object: \lstinline|PCA9535 LEDmux;|
    \item Initialize the object: usually a \lstinline|.begin()| method (PCA9535 requires more)
    \item Use the object: \lstinline|LEDmux.write(0xFF0E);|
\end{enumerate}

Now to begin showing some numbers on the 7-segment display.

\begin{enumerate}
    \item Run the script in Listing \ref{lst:leddispstart} to make sure it runs. Beware of copying
            multiple lines from a PDF to the Arduino IDE since it often adds in unwanted characters. 
            Also, underscores (\_) and some other symbols sometimes fail to transfer correctly. The 
            sketch is also available for download from the class Canvas site.
    \item Change the script so that it displays four consecutive numbers such as 1234. You can use
            any 4 consecutive, single digit numbers.
    \item Demonstrate your sketch working to a TA or instructor.
\end{enumerate}

\begin{lstlisting}[language=C++, caption={This listing is a starting point for driving the LED display. 
    This sketch may also be available on Canvas. Beware of copying out of PDFs since some characters
     (underscore for instance) come through garbled.},label={lst:leddispstart}]
    /* 20220907LEDsClass1.ino
    *  
    * Demo of LED display in class.
    * 
    * Seth McNeill
    * 2022 September 07
    */
   
   #include <PCA95x5.h>  // include library
   
   PCA9535 LEDmux;  // create instance (object) of library
   
   void setup() {
     Serial.begin(115200);
     delay(3000);
     Serial.println("Starting...");
   
     // initialize object
     Wire.begin();  // this must be done before LEDmux.attach
     LEDmux.attach(Wire, 0x21);  // 0x21 is the I2C address for the PCA9535 attached to the LEDs
     LEDmux.polarity(PCA95x5::Polarity::ORIGINAL_ALL);
     uint16_t mux_direction = 0x0010;  // mostly outputs (0), setting to 1 designates input
     LEDmux.direction(mux_direction);
   
     Serial.println("Everything setup");
   }
   
   
   void loop() {
     int delayTime = 0;
     // use object
     LEDmux.write(0x000F);  // required to remove ghosting on other digits
     LEDmux.write(0xFF0E);
     delay(delayTime); 
     LEDmux.write(0x000F);  // required to remove ghosting on other digits
     LEDmux.write(0x3F0D);
     delay(delayTime); 
     LEDmux.write(0x000F);  // required to remove ghosting on other digits
     LEDmux.write(0x3F0B);
     delay(delayTime); 
     LEDmux.write(0x000F);  // required to remove ghosting on other digits
     LEDmux.write(0x3F07);
     delay(delayTime); 
   }
\end{lstlisting}
Note the anti-ghosting lines in Listing \ref{lst:leddispstart}. This is because the two
bytes on the PCA9535 do not change simultaneously. One will change before the other 
giving a slight showing of the previous number before showing the new number. The fix 
is to turn off all four segments before writing the new number.
    
\subsection{Count}
Make a new sketch, based off your first sketch (meaning copy and paste it into the new sketch).
This new sketch should count up from 0 to 9 (or 0000 to 0009) incrementing once per second. Demonstrate
this counting to a TA/instructor and get it signed off.

\subsection{Extra credit: 2 pts}
Have your counting system count in increments of 1 using more than one digit (i.e. 
count from 00 to 99 if two digits). You must do the full count to 99, 999, or 9999,
not just count to 10. It should reset (overflow) back to zero and keep counting 
when it reaches the maximum. 
\subsection{Extra Credit Hints}
\begin{enumerate}
    \item Use the \lstinline|millis()| function to update count every second. The 
            following \lstinline|if| statement shows an example of how to do this: \\
            \lstinline|if(millis() - lastUpdate > updateInterval)| \\
            Be sure to update 
            \lstinline|lastUpdate| to the current value of \lstinline|millis()| inside
            the \lstinline|if| statement. \lstinline|lastUpdate| should be of type 
            \lstinline|unsigned long| and be declared as a \lstinline|static| variable 
            or in a global scope so that it's value persists between calls of the 
            \lstinline|loop()| function.
    \item I used an array for the segments required for each number something like \\
            \lstinline|nums[] = {0xAB,0x1C,...}|  // values not correct
    \item I also used an array to specify which digit is showing, something like \\
            \lstinline|digs[] = {0x12, 0x0E,...}| // values not correct
    \item These can be combined since \lstinline|nums[ii]| is the upper byte and 
            \lstinline|digs[jj]| is the lower byte of the number that needs to be 
            written to the PCA9535. \lstinline|nums[ii]| needs to be shifted to be 
            the upper byte using the \lstinline|<<| operator. For example: \\
            \lstinline@(nums[ii] << 8) | digs[jj]@ 
\end{enumerate}

\section{Turn In}
Turn in the following:
\begin{enumerate}
    \item Make sure that you have been signed off for both consecutive numbers and counting.
    \item A PDF of your first sketch that displays 4 consecutive numbers.
    \item A PDF of your second sketch that counts.
    \item .ino versions of both sketches.
    \item Fill out the end of lab quiz prior to leaving. Note that it includes asking you 
            for the output of the \lstinline$getIDs$ sketch. 
\end{enumerate}