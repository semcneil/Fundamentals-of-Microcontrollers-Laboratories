\chapter{Controls}
\chaplabel{controls}

\section{Purpose}
The goal of this lab is to play with PID control to get a feel for how it 
works on a real system.

\section{Laboratory}
\subsection{Getting started}
Example code is posted on Canvas as \lstinline@pid_dist_v0.5.ino@. Download and 
try running it. After pressing the right button it should try to maintain a 
distance from an object in front of the robot. The NeoPixels will be red when it 
is in PID mode, and green when it is in HALT mode. Turning the potentiometer
changes the target distance. In order to have the serial port work at the same 
time as the motors run do the following:
\begin{enumerate}
    \item Make sure both motor jumpers are set to BAT 
    \item Plug in the serial port 
    \item Turn on the battery switch.
\end{enumerate}

Make sure this code works as expected. 

\subsection{Proportional Control}
Set the gain for the integral (KI) and derivative (KD) terms to zero. 
Try different values for the proportional control. Can you turn proportional 
control into bang-bang control? How well does it work?

\subsection{Integral Control}
Set the gain for proportional (KP) and derivative (KD) control to zero. 
Try different values for integral control. How well does it run? Can you see 
the windup? What happens for large values of KI? What happens for small values of 
KI?

\subsection{Derivative Control}
Set the gain for proportional (KP) and integral (KI) control to zero.
Try different values for KD. What happens with large values? How about 
for small values?

\subsection{PID Calibration}
Based on the tests you have done so far. Try to find gain values for the 
full PID controller that work better than the defaults in the original file.
How can you tell if your gains are better than the original ones?

\subsection{IMU PID Control}
You have been supplied with a ruler to use as a ramp. Copy the PID file to 
a new one with IMU in the title. Change it to try to level itself by 
backing up the ramp. You will need to use your code for measuring angles 
from a previous lab.

Some other changes that need to be made:
\begin{enumerate}
    \item The distance PID controller is based on \lstinline@int@ values.
            The angle is a float, so the PID function and values need to be 
            updated accordingly.
    \item The error also needs to be changed to a float.
    \item Depending on the angle, the robot may not be able to get onto the
            ramp so you may need to start it on the ramp. Tip the ramp up
            and down to provide perturbances to the system.
    \item The axes that the code measures around need to be changed to 
            measure the front-to-back angle rather than the side-to-side 
            angle.
\end{enumerate}

Be sure to turn off the battery switch prior to putting the robot away. You do not need to turn in 
the distance PID sketch.

\section{Turn In}
Turn in the following:
\begin{enumerate}
    \item Have either the TA or the instructor sign-off on your lab
    \item A short writeup with the answers to the questions in the distance PID section. One per group is fine.
    \item A PDF of your ramp PID sketch.
    \item .ino versions of your ramp PID sketch.
    \item Fill out the end of lab quiz prior to leaving. Note that it includes asking you 
            for the output of the \lstinline$getIDs$ sketch. 
\end{enumerate}

\section{Resources}\label{sec:controlsresources}
\begin{enumerate}
    \item \href{https://github.com/stm32duino/LSM6DSOX/blob/main/src/LSM6DSOXSensor.h}{LSM6DSOX Library Header}
    \item \href{https://www.st.com/resource/en/datasheet/lsm6dsox.pdf}{LSM6DSOX Datasheet}
    \item PCB Schematic and Layout - see 
            \href{https://github.com/semcneil/Fundamentals-of-Microcontrollers-Manual}{class manual} 
            in the Arduino Startup $\rightarrow$ Schematics and PCB section
\end{enumerate}

