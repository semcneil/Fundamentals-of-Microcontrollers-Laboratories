\chapter{Distance, Motor, Servo}
\chaplabel{distMotorServo}

\section{Purpose}
The goal of this lab is to learn the distance sensor, driving DC motors, and driving 
servo motors.

\section{Main Requirements}
\subsection{Distance Calibration}
Using an example sketch for the distance sensor, create a calibration curve for the distance 
sensor out to it's maximum sensing distance using the lid of the shoebox as a target. The 
calibration curve should use at least 5 points and have the readout (0-765ish) on the x-axis
and actual, measured distance on the y-axis. Plot the calibration curve in your favorite data 
analysis software (Python, Matlab, Excel, etc.) Use the Pololu library for the VL6180 sensor 
and set SCALING to 3.

\subsection{DC Motors}
Write a sketch that drives your robot forward until the distance sensor senses something 
(you choose a reasonable threshold), and then turns to avoid it and continues on. Remember
that to drive the motors at different speeds, you can just use \lstinline@analogWrite@ on 
one pin and set the other pin to 0 (\lstinline@digitalWrite(0)@). The motor pins are listed 
as \lstinline@AIN1, AIN2, BIN1, BIN2@ which map to 0, 1, 8, and A2, respectively, on the 
schematic.

\subsection{Servo}
Write a sketch where the distance is measured every 10~degrees of servo movement. Plot the
data on the robot's TFT display. The \lstinline@Servo@ library is already installed and 
not far down in the Examples. Look at the Sweep example to get started.

% \section{Procedure}
% It is best to follow this outline to finish this lab most efficiently.
% \begin{enumerate}
%     \item 
% \end{enumerate}

\section{Turn In}
Turn in the following:
\begin{enumerate}
    \item A video of your board fulfilling the requirements in the Procedure section 
            that includes both of your faces and you saying your names.
    \item A PDF of your sketch.
    \item .ino versions of your sketch.
    \item The script/file you used to plot the data.
    \item Fill out the end of lab quiz prior to leaving. Note that it includes asking you 
            for the output of the \lstinline$getIDs$ sketch. 
\end{enumerate}

\section{Resources}\label{sec:distmotorservoresources}
\begin{enumerate}
    \item \href{https://www.st.com/resource/en/datasheet/vl6180.pdf}{VL6180 Distance sensor}
    \item \href{https://datasheet.lcsc.com/lcsc/2001060933_TMI-TMI8837_C478955.pdf}{TMI8837 Motor controller}
    \item PCB Schematic and Layout - see 
            \href{https://github.com/semcneil/Fundamentals-of-Microcontrollers-Manual}{class manual} 
            in the Arduino Startup $\rightarrow$ Schematics and PCB section
\end{enumerate}

