\chapter{Peak Detection}
\chaplabel{peakDetection}

% This lab didn't take very long


\section{Purpose}
The goal of this lab is to experiment with extrema detection by creating a sketch
that has the following two parts:
\begin{enumerate}
    \item Detect peaks/valleys in data 
    \item Do something when extrema are detected
\end{enumerate}

%\section{Main Requirements}


\section{Procedure}

\begin{enumerate}
    \item Install the PeakDetection library from:\\
          \href{https://github.com/leandcesar/PeakDetection}{https://github.com/leandcesar/PeakDetection}
    \item Click on the green Code button and select Download ZIP
    \item On the Arduino IDE click Sketch $\rightarrow$ Include Library $\rightarrow$ Add ZIP Library
    \item The example uses input from \lstinline@A0@, which is attached to SW1 on your board. 
            This should work for initial testing that the code runs.
    \item There is an example posted on Canvas that uses the potentiometer on your board, download
            and make sure this example runs too.
    \item Choose which sensor you plan to use (not SW1 or the potentiometer)
    \item Choose how you want the board to react to positive and negative extrema 
    \begin{enumerate}
        \item NOTE: Your board must respond to a peak/valley (extrema), not just a threshold situation.
        \item For example, if you are using a distance sensor, it must react to something
                coming closer and then getting further away. It cannot just react to something 
                getting closer like we did in the Distance, Motor, Servo lab.
        \item Your board must respond differently to a positive extrema (peak == 1) than a 
                negative extrema (peak == -1).
        \item We will be checking for the adaptation of your program to changing baselines. 
    \end{enumerate}
    \item Implement your choices of sensor and reactions
    \item Complete what is required to turn in
\end{enumerate}

\subsection{Possible Sensors}
\begin{enumerate}
    \item IMU (accelerometer and/or gyroscope)
    \begin{enumerate}
        \item React to waving board
        \item React to tipping board 
        \item React to impacts
    \end{enumerate}
    \item Proximity (look for peak, not threshold situation)
    \item Light
    \item Temperature
    \item Humidity
    \item Compass
\end{enumerate}

\subsection{Example Reactions}
You can probably come up with more ideas for reactions.
\begin{enumerate}
    \item Run buzzer with different \lstinline@tone@s 
    \item Move servo 
    \item Change motor speed/direction
    \item Change NeoPixels
    \item Change LED display
    \item Display relevant data on the TFT display
\end{enumerate}

\section{Turn In}
Turn in the following:
\begin{enumerate}
    \item Have either the TA or the instructor sign-off on your lab
    \item A PDF of your sketch.
    \item .ino versions of your sketch.
    \item Fill out the end of lab quiz prior to leaving. Note that it includes asking you 
            for the output of the \lstinline$getIDs$ sketch. 
\end{enumerate}

\section{Resources}\label{sec:peakdetectionresources}
\begin{enumerate}
    \item \href{https://stackoverflow.com/a/22640362}{StackOverflow post about this algorithm}
    \item PCB Schematic and Layout - see 
            \href{https://github.com/semcneil/Fundamentals-of-Microcontrollers-Manual}{class manual} 
            in the Arduino Startup $\rightarrow$ Schematics and PCB section
\end{enumerate}

